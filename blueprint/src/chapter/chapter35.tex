\chapter{The finite Kakeya problem}

Let $F$ be a finite field.

\begin{lemma}
  \label{ch35lemma1}
  Every nonzero polynomial $p(x) \in F[x_1, \dots, x_n]$ of degree $d$ has at
  most $dq^{n-1}$ roots in $F^n$.
\end{lemma}
\begin{proof}
  We use induction on $n$, with fact (1) above as the starting case $n = 1$. Let us split $p(x)$ into
  summands according to the powers of $x_n$,
\[
p(x) = g_0 + g_1 x_n + g_2 x_n^2 + \cdots + g_\ell x_n^\ell,
\]
where $g_i \in F[x_1, \dots, x_{n-1}]$ for $0 \leq i \leq \ell \leq d$, and $g_\ell$ is nonzero.
We write every $v \in F^n$ in the form $v = (a, b)$ with $a \in F^{n-1}$, $b \in F$, and
estimate the number of roots $p(a, b) = 0$.

\textbf{Case 1.} Roots $(a, b)$ with $g_\ell(a) = 0$.
Since $g_\ell \neq 0$ and $\deg g_\ell \leq d - \ell$, by induction the polynomial $g_\ell$ has at
most $(d - \ell) q^{n-2}$ roots in $F^{n-1}$, and for each $a$ there are at most $q$ different
choices for $b$, which gives at most $(d - \ell)q^{n-1}$ such roots for $p(x)$ in $F^n$.

\textbf{Case 2.} Roots $(a, b)$ with $g_\ell(a) \neq 0$.
Here $p(a, x_n) \in F[x_n]$ is not the zero polynomial in the single variable $x_n$, it has degree
$\ell$, and hence for each $a$ by (1) there are at most $\ell$ elements $b$ with $p(a, b) = 0$. Since the number of $a$'s is at most $q^{n-1}$ we get at most $\ell q^{n-1}$ roots for $p(x)$ in this way.

Summing the two cases gives at most
\[
(d - \ell)q^{n-1} + \ell q^{n-1} = dq^{n-1}
\]
roots for $p(x)$, as asserted.
\end{proof}

\begin{lemma}
  \label{ch35lemma2}
For every set $E \subseteq F^n$ of size $|E| < \binom{n+d}{d}$ there is a nonzero
polynomial $p(x) \in F[x_1, \dots, x_n]$ of degree at most $d$ that vanishes on $E$.
\end{lemma}
\begin{proof}
Consider the vector space $V_d$ of all polynomials in $F[x_1, \dots, x_n]$ of degree at most $d$.
A basis for $V_d$ is provided by the monomials $x_1^{s_1} \dots x_n^{s_n}$ with $\sum s_i \leq d$:
\[
1, x_1, \dots, x_n, x_1^2, x_1 x_2, \dots, x_1^3, \dots, x_n^d.
\]
The following pleasing argument shows that the number of monomials $x_1^{s_1} \dots x_n^{s_n}$ of
degree at most $d$ equals the binomial coefficient $\binom{n+d}{d}$. What we want to count is
the number of $n$-tuples $(s_1, \dots, s_n)$ of nonnegative integers with
$s_1 + \cdots + s_n \leq d$. To do this, we map every $n$-tuple $(s_1, \dots, s_n)$ to the
increasing sequence
\[
s_1 + 1 < s_1 + s_2 + 2 < \cdots < s_1 + \cdots + s_n + n,
\]
which determines an $n$-subset of $\{1, 2, \dots, d + n\}$. The map is bijective, so the number of
 monomials is $\binom{n+d}{d}$.

Next look at the vector space $F^E$ of all functions $f : E \to F$; it has dimension $|E|$,
which by assumption is less than $\binom{n+d}{d} = \dim V_d$. The evaluation
map $p(x) \mapsto (p(a))_{a \in E}$ from $V_d$ to $F^E$ is a linear map of vector spaces.
We conclude that it has a nonzero kernel,
containing as desired a nonzero polynomial that vanishes on $E$.
\end{proof}

\begin{theorem}[finite Kakeya problem]
  \label{kakeya}
  Let $K \subseteq F^n$ be a Kakeya set. Then
\[
|K| \geq \binom{|F| + n - 1}{n} \geq \frac{|F|^n}{n!}.
\]
\end{theorem}
\begin{proof}
  \uses{ch35lemma1, ch35lemma2}
  The second inequality is clear from the definition of binomial coefficients. For the first,
  set again $q = |F|$ and suppose for a contradiction that
\[
|K| < \binom{q + n - 1}{n} = \binom{n + q - 1}{q - 1}.
\]
By Lemma \ref{ch35lemma2} there exists a nonzero
polynomial $p(x) \in F[x_1, \dots, x_n]$ of degree $d \leq q - 1$ that
vanishes on $K$. Let us write
\[
p(x) = p_0(x) + p_1(x) + \cdots + p_d(x), \tag{1}
\]
where $p_i(x)$ is the sum of the monomials of degree $i$; in particular, $p_d(x)$ is nonzero.
Since $p(x)$ vanishes on the nonempty set $K$, we have $d > 0$.
Take any $v \in F^n \setminus \{0\}$. By the Kakeya property for this $v$ there
exists a $w \in F^n$ such that
\[
p(w + tv) = 0 \quad \text{for all} \ t \in F.
\]
Here comes the trick: Consider $p(w + tv)$ as a polynomial in the single variable $t$.
It has degree at most $d \leq q - 1$ but vanishes on all $q$ points of $F$,
whence $p(w + tv)$ is the zero polynomial in $t$. Looking at (1) above we see that the
coefficient of $t^d$ in $p(w + tv)$ is precisely $p_d(v)$, which must therefore be 0.
But $v \in F^n \setminus \{0\}$ was arbitrary and $p_d(0) = 0$ since $d > 0$,
and we conclude that $p_d(x)$ vanishes on all of $F^n$. Since
\[
dq^{n-1} \leq (q - 1)q^{n-1} < q^n,
\]
Lemma \ref{ch35lemma1}, however, tells us that $p_d(x)$ must then be the zero polynomial —
contradiction and end of the proof.

\end{proof}
