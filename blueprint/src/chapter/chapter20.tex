\chapter{In praise of inequalities}
\label{chapter20}
\begin{theorem}
  \label{ch20theoremI}
  \lean{cauchy_schwarz_inequality}
  \leanok
Let $\langle a, b \rangle$ be an inner product on a real vector space $V$ (with the norm $|a|^2 := \langle a, a \rangle$). Then
\[
\langle a, b \rangle^2 \leq |a|^2 |b|^2
\]
holds for all vectors $a, b \in V$, with equality if and only if $a$ and $b$ are linearly dependent.
\end{theorem}
\begin{proof}
  \leanok
The following (folklore) proof is probably the shortest. Consider the quadratic function
\[
|x a + b|^2 = x^2 |a|^2 + 2x \langle a, b \rangle + |b|^2
\]
in the variable $x$. We may assume $a \neq 0$. If $b = \lambda a$, then clearly
\[
\langle a, b \rangle^2 = |a|^2 |b|^2.
\]
If, on the other hand, $a$ and $b$ are linearly independent, then $|x a + b|^2 > 0$ for all $x$,
and thus the discriminant $\langle a, b \rangle^2 - |a|^2 |b|^2$ is less than 0.
\end{proof}

\begin{theorem}[First proof]
  \label{ch20theoremIIproof1}
  \lean{harmonic_geometric_arithmetic₁}
  Let $a_1, \dots a_n$ be positive real numbers, then
  \[
  \frac{n}{\frac{1}{a_1}+\dots+\frac{1}{n_n}} \le
    \sqrt[n]{a_1a_2\dots a_n} \le
    \frac{a_1\dots a_n}{n}
  \]
 with equality in both cases if and only if all $a_i$'s are equal.
\end{theorem}
 \begin{proof}
  TODO
\end{proof}

\begin{theorem}[Another Proof]
  \label{ch20theoremIIproof2}
  \lean{harmonic_geometric_arithmetic₂ }
  Let $a_1, \dots a_n$ be positive real numbers, then
  \[
  \frac{n}{\frac{1}{a_1}+\dots+\frac{1}{n_n}} \le
    \sqrt[n]{a_1a_2\dots a_n} \le
    \frac{a_1\dots a_n}{n}
  \]
 with equality in both cases if and only if all $a_i$'s are equal.
\end{theorem}
 \begin{proof}
  TODO
\end{proof}

\begin{theorem}[Still another Proof]
  \label{ch20theoremIIproof3}
  \lean{harmonic_geometric_arithmetic₃}
  Let $a_1, \dots a_n$ be positive real numbers, then
  \[
  \frac{n}{\frac{1}{a_1}+\dots+\frac{1}{n_n}} \le
    \sqrt[n]{a_1a_2\dots a_n} \le
    \frac{a_1\dots a_n}{n}
  \]
 with equality in both cases if and only if all $a_i$'s are equal.
\end{theorem}
 \begin{proof}
  TODO
\end{proof}

\begin{theorem}
  \label{ch20theorem1}
  Suppose all roots fo the polynomial
  $x^n + a_{n - 1}x^{n - 1} + \dots +a_0$ are real.
  Then the roots of are contained in the interval with the endpoints
  \[
  -\frac{n_{n-1}}{n}\pm\frac{n - 1}{n}\sqrt{a^n_{n - 1} - \frac{2n}{n - 1}a_{n - 2}}.
  \]
\end{theorem}
\begin{proof}
  TODO
\end{proof}



\begin{theorem}
  \label{ch20theorem2} Let $f(x)$ be a real polynomial of degree $n \ge 2$ with only real
  roots, such that $f(x)> 0$ for $ -1 < x < 1$ amd $f(-1) = f(1) = 0$. Then
  \[
  \frac{2}{3}T \le A \le \frac{2}{3}R,
  \]
  and equality holds in both cases only for $n=2$.
\end{theorem}
\begin{proof}
  TODO
\end{proof}

\begin{theorem}
  \label{ch20theorem3proof1}
  \lean{mantel}
  Suppose $G$ is a graph on $n$ vertices without triangles. Then $G$
  has at most $\frac{n^2}{4}$ edges, and equality holds only when $n$ is even and $G$ is the
  complete bipartite graph $K_{n/2, n/2}$.
\end{theorem}
\begin{proof}
  This proof, using Cauchy's inequality, is due to Mantel.
  Let $V = \{1, \dots, n\}$ be the vertex set and $E$ the edge set of $G$.
  By $d_i$ we denote the degree of $i$, hence $\sum_{i \in V} d_i = 2|E|$
  (see chapter \ref{chapter28}). Suppose $ij$ is an edge.
  Since $G$ has no triangles, we find $d_i + d_j \leq n$ since no vertex is a
  neighbor of both $i$ and $j$.

  It follows that
  \[
  \sum_{ij \in E} (d_i + d_j) \leq n|E|.
  \]
  Note that $d_i$ appears exactly $d_i$ times in the sum, so we get
  \[
  n|E| \geq \sum_{ij \in E} (d_i + d_j) = \sum_{i \in V} d_i^2,
  \]
  and hence with Cauchy's inequality applied to the vectors $(d_1, \dots, d_n)$ and $(1, \dots, 1)$,
  \[
  n|E| \geq \sum_{i \in V} d_i^2 \geq \frac{\left( \sum d_i \right)^2}{n} = \frac{4|E|^2}{n},
  \]
  and the result follows. In the case of equality we find $d_i = d_j$ for all $i, j$, and further
  $d_i = \frac{n}{2}$ (since $d_i + d_j = n$). Since $G$ is triangle-free, $G = K_{n/2, n/2}$ is
  immediately seen from this.

\end{proof}

\begin{theorem}
  \label{ch20theorem3proof2}
  Suppose $G$ is a graph on $n$ vertices without triangles. Then $G$
  has at most $\frac{n^2}{4}$ edges, and equality holds only when $n$ is even and $G$ is the
  complete bipartite graph $K_{n/2, n/2}$.
\end{theorem}
\begin{proof}
  TODO
\end{proof}
