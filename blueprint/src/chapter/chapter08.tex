\chapter{Some irrational numbers}


\begin{theorem}
  \label{e_irrational}
  \lean{book.irrational.e_irrational}
  \leanok
  \(e\) is irrational
\end{theorem}
\begin{proof}
  TODO
\end{proof}

\begin{theorem}
  \label{e_pow_2_irrational}
  \lean{book.irrational.e_pow_2_irrational}
  \leanok
  \(e ^ 2\) is irrational
\end{theorem}
\begin{proof}
  \uses{e_irrational}
  TODO
\end{proof}


\begin{theorem}[Little Lemma]
  \label{little_lemma}
  \lean{book.irrational.little_lemma}
  \leanok
  TODO
\end{theorem}
\begin{proof}
  TODO
\end{proof}

\begin{theorem}
  \label{e_pow_4_irrational}
  \lean{book.irrational.e_pow_4_irrational}
  \leanok
  \(e ^ 4\) is irrational
\end{theorem}
\begin{proof}
  \uses{little_lemma, e_pow_2_irrational}
  TODO
\end{proof}

\begin{lemma}
  \label{lem_aux_i}
  \lean{book.irrational.lem_aux_i}
  \leanok
  TODO
\end{lemma}
\begin{proof}
  TODO
\end{proof}

\begin{lemma}
  \label{lem_aux_ii}
  \lean{book.irrational.lem_aux_ii}
  \leanok
  TODO
\end{lemma}
\begin{proof}
  TODO
\end{proof}


\begin{lemma}
  \label{lem_aux_iii}
  \lean{book.irrational.lem_aux_iii}
  \leanok
  TODO
\end{lemma}
\begin{proof}
  TODO
\end{proof}

\begin{theorem}
  \label{e_pow_irrational}
  \label{book.irrational.Theorem_1}
  \leanok
  \(e^r\) is irrational for every \(r\in\mathbb{Q}\setminus\{0\}\).
\end{theorem}
\begin{proof}
  \uses{lem_aux_i, lem_aux_ii, lem_aux_iii}
  TODO
\end{proof}


\begin{theorem}
  \label{pi_pow_2_irrational}
  \label{book.irrational.Theorem_2}
  \leanok
  \(\pi^2\) is irrational.
\end{theorem}
\begin{proof}
  \uses{lem_aux_ii, lem_aux_iii}
  TODO
\end{proof}

\begin{theorem}
  \label{arccos_irrational}
  \label{book.irrational.Theorem_3}
  \leanok
  For every odd integer \(n \ge 3\), the number
  \[
  A(n) := \frac{1}{\pi}\arccos\left(\frac{1}{\sqrt{n}}\right)
  \]
  is irrational.
\end{theorem}
\begin{proof}
  \uses{pi_pow_2_irrational}
  TODO
\end{proof}
