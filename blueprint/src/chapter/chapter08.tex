\chapter{Some irrational numbers}


\begin{theorem}
  \label{e_irrational}
  \lean{book.irrational.e_irrational}
  \leanok
  $e$ is irrational.
\end{theorem}	
\begin{proof}
  To start with, it is rather easy to see (as did Fourier in 1815) that $e = \sum_{k \ge 0} \frac{1}{k!}$ is irrational.
  Indeed, if we had $e = \frac{a}{b}$ for integers $a$ and $b > 0$, then we would get
  \[
  n! b e = n! a
  \]
  for \emph{every} $n \ge 0$.
  But this cannot be true, because on the right-hand side we have an integer, while the left-hand side with
  \[
  e = 1 + \frac{1}{1!} + \frac{1}{2!} + \dots + \frac{1}{n!} + \frac{1}{(n+1)!} + \frac{1}{(n+2)!} + \frac{1}{(n+3)!} + \dots
  \]
  decomposes into an integral part
  \[
  b n! \left( 1 + \frac{1}{1!} + \frac{1}{2!} + \dots + \frac{1}{n!} \right)
  \]
  and a second part
  \[
  b \left( \frac{1}{n+1} + \frac{1}{(n+1)(n+2)} + \frac{1}{(n+1)(n+2)(n+3)} + \dots \right)
  \]
  which is \emph{approximately} $\frac{b}{n}$, so that for large $n$ it certainly cannot be integral:
  It is larger than $\frac{b}{n+1}$ and smaller than $\frac{b}{n}$, as one can see from a comparison with a geometric series:
  \begin{align*}
  \frac{1}{n+1} &< \frac{1}{n+1} + \frac{1}{(n+1)(n+2)} + \frac{1}{(n+1)(n+2)(n+3)} + \dots \\
  &< \frac{1}{n+1} + \frac{1}{(n+1)^2} + \frac{1}{(n+1)^3} + \dots = \frac{1}{n}.
  \end{align*}
\end{proof}

\begin{theorem}
  \label{e_pow_2_irrational}
  \lean{book.irrational.e_pow_2_irrational}
  \leanok
  $e^2$ is irrational.
\end{theorem}
\begin{proof}
  \uses{e_irrational}
  Now one might be led to think that this simple multiply-by-$n!$ trick is not sufficient to show that $e^2$ is irrational.
  This is a stronger statement:
  $\sqrt{2}$ is an example of a number which is irrational, but whose square is not.
  From John Cosgrave we have learned that with two nice ideas/observations (let's call them ``tricks'') one can get two steps further nevertheless:
  Each of the tricks is sufficient to show that $e^2$ is irrational, the combination of both of them even yields the same for $e^4$.
  The first trick may be found in a one page paper by J. Liouville from 1840 --- and the second one in a two page ``addendum''
  which Liouville published on the next two journal pages.

  Why is $e^2$ irrational?
  What can we derive from $e^2 = \frac{a}{b}$?
  According to Liouville we should write this as
  \[
  b e = a e^{-1},
  \]
  substitute the series
  \[
  e = 1 + \frac{1}{1} + \frac{1}{2} + \frac{1}{6} + \frac{1}{24} + \frac{1}{120} + \cdots
  \]
  and
  \[
  e^{-1} = 1 - \frac{1}{1} + \frac{1}{2} - \frac{1}{6} + \frac{1}{24} - \frac{1}{120} \pm \cdots,
  \]
  and then multiply by $n!$, for a sufficiently large even $n$.
  Then we see that $n! b e$ is nearly integral:
  \[
  n! b \left( 1 + \frac{1}{1} + \frac{1}{2} + \frac{1}{6} + \dots + \frac{1}{n!} \right)
  \]
  is an integer, and the rest
  \[
  n! b \left( \frac{1}{(n+1)!} + \frac{1}{(n+2)!} + \dots \right)
  \]
  is approximately $\frac{b}{n}$:
  It is larger than $\frac{b}{n+1}$ but smaller than $\frac{b}{n}$, as we have seen above.
  
  At the same time $n! a e^{-1}$ is nearly integral as well:
  Again we get a large integral part, and then a rest
  \[
  (-1)^{n+1} n! a \left( \frac{1}{(n+1)!} - \frac{1}{(n+2)!} + \frac{1}{(n+3)!} \mp \cdots \right),
  \]
  and this is approximately $(-1)^{n+1} \frac{a}{n}$.
  More precisely:
  for even $n$ the rest is larger than $-\frac{a}{n}$, but smaller than
  \[
  -a \left( \frac{1}{n+1} - \frac{1}{(n+1)^2} - \frac{1}{(n+1)^3} - \cdots \right) = - \frac{a}{n+1} \left( 1 - \frac{1}{n} \right) < 0.
  \]
  But this cannot be true, since for large even $n$ it would imply that $n! a e^{-1}$ is just a bit smaller than an integer, while $n! b e$ is a bit larger than an integer, so $n! a e^{-1} = n! b e$ cannot hold.
\end{proof}


\begin{theorem}[Little Lemma]
  \label{little_lemma}
  \lean{book.irrational.little_lemma}
  \leanok
  For any $n \ge 1$ the integer $n!$ contains the prime factor $2$ at most $n-1$ times --- with equality if (and only if) $n$ is a power of two, $n = 2^m$.
\end{theorem}
\begin{proof}
  This lemma is not hard to show:
  $\lfloor \frac{n}{2} \rfloor$ of the factors of $n!$ are even, $\lfloor \frac{n}{4} \rfloor$ of them are divisible by $4$, and so on.
  So if $2^k$ is the largest power of two which satisfies $2^k \le n$, then $n!$ contains the prime factor $2$ exactly
  \[
  \left\lfloor \frac{n}{2} \right\rfloor + \left\lfloor \frac{n}{4} \right\rfloor + \dots + \left\lfloor \frac{n}{2^k} \right\rfloor \le \frac{n}{2} + \frac{n}{4} + \dots + \frac{n}{2^k} = n \left( 1 - \frac{1}{2^k} \right) \le n-1
  \]
  times, with equality in both inequalities exactly if $n = 2^k$.
\end{proof}

\begin{theorem}
  \label{e_pow_4_irrational}
  \lean{book.irrational.e_pow_4_irrational}
  \leanok
  $e^4$ is irrational.
\end{theorem}
\begin{proof}
  \uses{little_lemma, e_pow_2_irrational}
  In order to show that $e^4$ is irrational, we now courageously assume that $e^4 = \frac{a}{b}$ were rational, and write this as
  \[
  b e^2 = a e^{-2}.
  \]
  We could now try to multiply this by $n!$ for some large $n$, and collect the non-integral summands, but this leads to nothing useful:
  The sum of the remaining terms on the left-hand side will be approximately $b \frac{2^{n+1}}{n}$,
  on the right side $(-1)^{n+1} a \frac{2^{n+1}}{n}$, and both will be very large if $n$ gets large.
  
  So one has to examine the situation a bit more carefully, and make two little adjustments to the strategy:
  First we will not take an \emph{arbitrary} large $n$, but a large power of two, $n = 2^m$;
  and secondly we will not multiply by $n!$, but by $\frac{n!}{2^{n-1}}$.
  Then we need the little lemma \ref{little_lemma}, a special case of Legendre's theorem (see page 10).

  Let's get back to $b e^2 = a e^{-2}$.
  We are looking at
  \begin{equation} \label{eq:e4_irrational}
  b \frac{n!}{2^{n-1}} e^2 = a \frac{n!}{2^{n-1}} e^{-2}
  \end{equation}
  and substitute the series
  \[
  e^2 = 1 + \frac{2}{1} + \frac{4}{2} + \frac{8}{6} + \dots + \frac{2^r}{r!} + \dots
  \]
  and
  \[
  e^{-2} = 1 - \frac{2}{1} + \frac{4}{2} - \frac{8}{6} \pm \dots + (-1)^r \frac{2^r}{r!} + \dots
  \]
  For $r \le n$ we get integral summands on both sides, namely
  \[
  b \frac{n!}{2^{n-1}} \frac{2^r}{r!} \quad \text{resp.} \quad (-1)^r a \frac{n!}{2^{n-1}} \frac{2^r}{r!},
  \]
  where for $r > 0$ the denominator $r!$ contains the prime factor $2$ at most $r-1$ times, while $n!$ contains it \emph{exactly} $n-1$ times.
  (So for $r > 0$ the summands are even.)
  
  And since $n$ is even (we assume that $n = 2^m$), the series that we get for $r \ge n+1$ are
  \[
  2b \left( \frac{2}{n+1} + \frac{4}{(n+1)(n+2)} + \frac{8}{(n+1)(n+2)(n+3)} + \dots \right)
  \]
  resp.
  \[
  2a \left( - \frac{2}{n+1} + \frac{4}{(n+1)(n+2)} - \frac{8}{(n+1)(n+2)(n+3)} \pm \dots \right).
  \]
  These series will for large $n$ be roughly $\frac{4b}{n}$ resp. $-\frac{4a}{n}$, as one sees again by comparison with geometric series.
  For large $n = 2^m$ this means that the left-hand side of \eqref{eq:e4_irrational} is a bit larger than an integer,
  while the right-hand side is a bit smaller --- contradiction!
\end{proof}

\begin{lemma}
  \label{lem_aux_i}
  \lean{book.irrational.lem_aux_i}
  \leanok
  \upshape For some fixed $n \ge 1$, let
  \[
  f(x) = \frac{x^n (1-x)^n}{n!}.
  \]
  \begin{enumerate}
    \item[(i)] \itshape The function $f(x)$ is a polynomial of the form $f(x) = \frac{1}{n!} \sum_{i=n}^{2n} c_i x^i$, where the coefficients $c_i$ are integers.
  \end{enumerate}
\end{lemma}
\begin{proof}
  Part (i) is clear.
\end{proof}

\begin{lemma}
  \label{lem_aux_ii}
  \lean{book.irrational.lem_aux_ii}
  \leanok
  \begin{enumerate}
  	\item[(ii)] For $0 < x < 1$ we have $0 < f(x) < \frac{1}{n!}$.
  \end{enumerate}
\end{lemma}
\begin{proof}
  Part (ii) is also clear.
\end{proof}


\begin{lemma}
  \label{lem_aux_iii}
  \lean{book.irrational.lem_aux_iii}
  \leanok
  \begin{enumerate}
  	\item[(iii)] The derivatives $f^{(k)}(0)$ and $f^{(k)}(1)$ are integers for all $k \ge 0$.
  \end{enumerate}
\end{lemma}
\begin{proof}
  For (iii) note that by (i) the $k$-th derivative $f^{(k)}$ vanishes at $x=0$ unless $n \le k \le 2n$,
  and in this range $f^{(k)}(0) = \frac{k!}{n!} c_k$ is an integer.
  From $f(x) = f(1-x)$ we get $f'(x) = (-1) f'(1-x)$ for all $x$,
  and hence $f^{(k)}(1) = (-1)^k f^{(k)}(0)$, which is an integer.
\end{proof}

\begin{theorem}
  \label{e_pow_irrational}
  \label{book.irrational.Theorem_1}
  \leanok
  $e^r$ is irrational for every $r \in \mathbb{Q} \setminus \{0\}$.
\end{theorem}
\begin{proof}
  \uses{lem_aux_i, lem_aux_ii, lem_aux_iii}
  It suffices to show that $e^s$ cannot be rational for a positive integer $s$ (if $e^{\frac{s}{t}}$ were rational, then $\left(e^{\frac{s}{t}}\right)^t = e^s$ would be rational, too).
  Assume that $e^s = \frac{a}{b}$ for integers $a, b > 0$, and let $n$ be so large that $n! > a s^{2n+1}$.
  Put
  \[
  F(x) := s^{2n} f(x) - s^{2n-1} f'(x) + s^{2n-2} f''(x) \mp \dots + f^{(2n)}(x),
  \]
  where $f(x)$ is the function of the lemma.
  $F(x)$ may also be written as an infinite sum
  \[
  F(x) = s^{2n} f(x) - s^{2n-1} f'(x) + s^{2n-2} f''(x) \mp \cdots,
  \]
  since the higher derivatives $f^{(k)}(x)$, for $k > 2n$, vanish.
  From this we see that the polynomial $F(x)$ satisfies the identity
  \[
  F'(x) = -s F(x) + s^{2n+1} f(x).
  \]
  Thus differentiation yields
  \[
  \frac{d}{dx} [e^{sx} F(x)] = s e^{sx} F(x) + e^{sx} F'(x) = s^{2n+1} e^{sx} f(x)
  \]
  and hence
  \[
  N := b \int_0^1 s^{2n+1} e^{sx} f(x) dx = b [e^{sx} F(x)]_0^1 = a F(1) - b F(0).
  \]
  This is an integer, since part (iii) of the lemma implies that $F(0)$ and $F(1)$ are integers.
  However, part (ii) of the lemma yields estimates for the size of $N$ from below and from above,
  \[
  0 < N = b \int_0^1 s^{2n+1} e^{sx} f(x) dx < b s^{2n+1} e^s \frac{1}{n!} = \frac{a s^{2n+1}}{n!} < 1,
  \]
  which shows that $N$ cannot be an integer: contradiction.
\end{proof}


\begin{theorem}
  \label{pi_pow_2_irrational}
  \label{book.irrational.Theorem_2}
  \leanok
  $\pi^2$ is irrational.
\end{theorem}
\begin{proof}
  \uses{lem_aux_ii, lem_aux_iii}
  Assume that $\pi^2 = \frac{a}{b}$ for integers $a, b > 0$.
  We now use the polynomial
  \[
  F(x) := b^n \left( \pi^{2n} f(x) - \pi^{2n-2} f^{(2)}(x) + \pi^{2n-4} f^{(4)}(x) \mp \cdots \right),
  \]
  which satisfies $F''(x) = -\pi^2 F(x) + b^n \pi^{2n+2} f(x)$.
  
  From part (iii) of the lemma we get that $F(0)$ and $F(1)$ are integers.
  Elementary differentiation rules yield
  \begin{align*}
  \frac{d}{dx} [F'(x) \sin \pi x - \pi F(x) \cos \pi x] &= \left(F''(x) + \pi^2 F(x)\right) \sin \pi x \\
  &= b^n \pi^{2n+2} f(x) \sin \pi x \\
  &= \pi^2 a^n f(x) \sin \pi x,
  \end{align*}
  and thus we obtain
  \begin{align*}
  N := \pi \int_0^1 a^n f(x) \sin \pi x \, dx &= \left[ \frac{1}{\pi} F'(x) \sin \pi x - F(x) \cos \pi x \right]_0^1 \\
  &= F(0) + F(1),
  \end{align*}
  which is an integer.
  Furthermore $N$ is positive since it is defined as the integral of a function that is positive (except on the boundary).
  However, if we choose $n$ so large that $\frac{\pi a^n}{n!} < 1$, then from part (ii) of the lemma we obtain
  \[
  0 < N = \pi \int_0^1 a^n f(x) \sin \pi x dx < \frac{\pi a^n}{n!} < 1,
  \]
  a contradiction.
\end{proof}

\begin{theorem}
  \label{arccos_irrational}
  \label{book.irrational.Theorem_3}
  \leanok
  For every odd integer \(n \ge 3\), the number
  \[
  A(n) := \frac{1}{\pi} \arccos \left(\frac{1}{\sqrt{n}}\right)
  \]
  is irrational.
\end{theorem}
\begin{proof}
  \uses{pi_pow_2_irrational}
  We use the addition theorem
  \[
  \cos \alpha + \cos \beta = 2 \cos \frac{\alpha+\beta}{2} \cos \frac{\alpha-\beta}{2}
  \]
  from elementary trigonometry, which for $\alpha = (k+1)\varphi$ and $\beta = (k-1)\varphi$ yields
  \begin{equation} \label{eq:cos_add}
  \cos (k+1)\varphi = 2 \cos \varphi \cos k\varphi - \cos (k-1)\varphi.
  \end{equation}
  For the angle $\varphi_n = \arccos \left(\frac{1}{\sqrt{n}}\right)$, which is defined by $\cos \varphi_n = \frac{1}{\sqrt{n}}$
  and $0 \le \varphi_n \le \pi$, this yields representations of the form
  \[
  \cos k\varphi_n = \frac{A_k}{\sqrt{n}^k},
  \]
  where $A_k$ is an integer that is not divisible by $n$, for all $k \ge 0$.
  In fact, we have such a representation for $k=0, 1$ with $A_0 = A_1 = 1$, and by induction on $k$ using \eqref{eq:cos_add} we get for $k \ge 1$
  \[
  \cos (k+1)\varphi_n = 2 \frac{1}{\sqrt{n}} \frac{A_k}{\sqrt{n}^k} - \frac{A_{k-1}}{\sqrt{n}^{k-1}} = \frac{2A_k - n A_{k-1}}{\sqrt{n}^{k+1}}.
  \]
  Thus we obtain $A_{k+1} = 2A_k - n A_{k-1}$.
  If $n \ge 3$ is odd, and $A_k$ is not divisible by $n$, then we find that $A_{k+1}$ cannot be divisible by $n$, either.

  Now assume that
  \[
  A(n) = \frac{1}{\pi} \varphi_n = \frac{k}{\ell}
  \]
  is rational (with integers $k, \ell > 0$).
  Then $\ell \varphi_n = k \pi$ yields
  \[
  \pm 1 = \cos k \pi = \frac{A_\ell}{\sqrt{n}^\ell}.
  \]
  Thus $\sqrt{n}^\ell = \pm A_\ell$ is an integer, with $\ell \ge 2$, and hence $n | \sqrt{n}^\ell$.
  With $\sqrt{n}^\ell | A_\ell$ we find that $n$ divides $A_\ell$, a contradiction.
\end{proof}
