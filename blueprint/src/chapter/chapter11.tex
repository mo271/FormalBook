\chapter{Lines in the plane and decompositions of graphs}

\begin{theorem}
  \label{ch11_theorem1}
  In any configuration of \(n\) points in the plane, not all on a line,
  there is a line which contains exactly two of the points.
\end{theorem}
\begin{proof}
  TODO
\end{proof}

\begin{theorem}
  \label{ch11_theorem2}
  Let \(P\) be a set of \(n\ge 3\) points in the plane, not all on a line.
  Then the set \(\mathcal{L}\) of lines passing through at least two points contains at least
  \(n\) lines.
\end{theorem}
\begin{proof}
  TODO
\end{proof}

\begin{theorem}
  \label{ch11_theorem3}
  Let \(X\) be a set of \(n\ge 3\) elements, and let \(A_1, \dots, A_m\) be
  proper subsets of \(X\), such that every pair of elements of \(X\) is contained in
  precisely one set \(A_i\). Then \(m\ge n\) holds.
\end{theorem}
\begin{proof}
  TODO
\end{proof}

\begin{theorem}
  \label{ch11_theorem4}
  If \(K_n\) is decomposed into complete bipartite subgraphs
  \(H_1, \dots, H_m\), then \(m \ge n - 1\).
\end{theorem}
\begin{proof}
  TODO
\end{proof}
