\chapter{Three applications of Euler's formula}

\begin{theorem}[Euler's formula]
  \label{euler_formula}
  If \(G\) is a connected plane graph with \(n\) vertices, \(e\) edges and \(f\) faces, then
  \[
  n - e + f = 2.
  \]
\end{theorem}
\begin{proof}
  TODO
\end{proof}


\begin{proposition}
  \label{euler_consequence_a}
  Let \(G\) be any simple plane graph with \(n>2\) vertices.
  Then \(G\) has at most \(3*n - 6\) edges.
\end{proposition}
\begin{proof}
  TODO
  \uses{euler_formula}
\end{proof}

\begin{proposition}
  \label{euler_consequence_b}
  Let \(G\) be any simple plane graph with \(n>2\) vertices.
  Then \(G\) has a vertex of degree at most \(5\).
\end{proposition}
\begin{proof}
  TODO
  \uses{euler_consequence_a}
\end{proof}

\begin{proposition}
  \label{euler_consequence_c}
  Let \(G\) be any simple plane graph with \(n>2\) vertices.
  If the edges of \(G\) are two-colored, then there is a vertex of \(G\) with at most two
  color-changes in the cyclic order of the edges around the vertex.
\end{proposition}
\begin{proof}
  TODO
  \uses{euler_formula}
\end{proof}

\begin{theorem}[Sylvester-Gallai]
  \label{sylvester_gallai2}
  Given any set of \(n \ge 3\) points in the
  plane, not all on one line, there is always a line that contains exactly two
  of the points.
\end{theorem}
\begin{proof}
  TODO
  \uses{euler_consequence_b}
\end{proof}

\begin{theorem}[Monochromatic lines]
  \label{monochromatic_lines}
  Given any finite configuration of ``black'' and ``white'' points
  in the plane, not all on one line, there is always a ``monochromatic'' line:
  a line that contains at least two points of one color and none of the other.
\end{theorem}
\begin{proof}
  TODO
  \uses{euler_consequence_c}
\end{proof}

\begin{lemma}
  \label{pick_lemma}
  Every elementary triangle \(\Delta = \conv\{p_0, p_1, p_2\}\subset \mathbb{R}^2\) has area
  \(A(\Delta)=12\)
\end{lemma}
\begin{proof}
  TODO
\end{proof}


\begin{theorem}[Pick's theorem]
  \label{pick_theorem}
  The area of any (not necessarily convex) polygon \(Q\subset\mathbb{R}^2\) with
  integral vertices is given by
  \[
  A(Q) = n_{int} + \frac{1}{2}n_{bd} - 1
  \]
  where \(n_{int}\) and \(n_{bd}\) are the numbers of integral points in the interior
  respectively on the boundary of \(Q\).
\end{theorem}
\begin{proof}
  \uses{pick_lemma}
  \uses{euler_formula}
  TODO
\end{proof}
