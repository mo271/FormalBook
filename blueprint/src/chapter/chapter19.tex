\chapter{Sets, functions, and the continuum hypothesis}

\begin{theorem}
  \label{ch19theorem1}
  The set of \(\mathbb{Q}\) of rational numbers is countable.
\end{theorem}
\begin{proof}
  TODO
\end{proof}

\begin{theorem}
  \label{ch19theorem2}
  The set \(\mathbb{R}\) of real numbers is \emph{not} countable
\end{theorem}
\begin{proof}
  TODO
\end{proof}

\begin{theorem}
  \label{ch19theorem3}
  The set  \(\mathbb{R}^2\) of all ordered pairs of real numbers (that is, the
  real plane) has the same size as \(\mathbb{R}\).
\end{theorem}
\begin{proof}
  TODO
\end{proof}

\begin{theorem}
  \label{ch19theorem4}
  If each of two sets \(M\) and \(N\) can be mapped injectively into
  the other, then there is a bijection from \(M\) to \(N\), that is \(|M| = |N|\).
\end{theorem}
\begin{proof}
  TODO
\end{proof}

\begin{theorem}
  \label{ch19theorem5}
  If \(c > \aleph_1\), then every family \(\{f_\alpha\}\) satisfying \((P_0)\) is countable.
  If, on the other hand, \(c = \aleph_1\), then there exists some family \(\{f_\alpha\}\) with
  property \(P_0\) which has size \(c\).
\end{theorem}
\begin{proof}
  TODO
\end{proof}

\section*{Appendix} On cardinal and ordinal numbers

\begin{proposition}
  \label{ch19proposition1}
  Let \(\mu\) be an ordinal number and denote by \(W_\mu\) the set of
  ordinal numbers smaller than \(\mu\). Then the following holds:
  \begin{enumerate}
    \item The elements of \(W_\mu\) are pairwise comparable.
    \item If we order \(W_\mu\) according to their magnitude, then \(W_\mu\) is well-ordered and
    has ordinal number \(\mu\).
  \end{enumerate}
\end{proposition}
\begin{proof}
  TODO
\end{proof}

\begin{proposition}
  \label{ch19proposition2}
  Any two ordinal numbers \(\mu\) and \(\nu\) satisfy precisely one of
  the relations \(\mu < \nu\), \(\mu = \nu\), or \(\mu > \nu\).
\end{proposition}
\begin{proof}
  TODO
\end{proof}

\begin{proposition}
  \label{ch19proposition3}
  Every set of ordinal numbers (ordered according to magnitude) is well-ordered.
\end{proposition}
\begin{proof}
  TODO
\end{proof}

\begin{proposition}
  \label{ch19proposition4}
  For every cardinal number \(\mathfrak{m}\), there is a definite next larger cardinal number.
\end{proposition}
\begin{proof}
  TODO
\end{proof}

\begin{proposition}
  \label{ch19proposition5}
  Let the infinite set \(M\) have cardinality \(\mathfrak{m}\), and let \(M\) be well ordered
  according to the initial ordinal number \(\omega_{\mathfrak{m}}\). Then \(M\) has no last element.
\end{proposition}
\begin{proof}
  Indeed, if \(M\) had a last element \(m\), then the segment \(M_m\) would have an
  ordinal number \(\mu < \omega_{\mathfrak{m}}\) with \(|\mu| = \mathfrak{m}\),
  contradicting the definition of \(\omega_{\mathfrak{m}}\).
\end{proof}

\begin{proposition}
  \label{ch19proposition6}
  Suppose \(\{A_\alpha\}\) is a family of size \(\mathfrak{m}\) of countable sets \(A_\alpha\),
  where \(\mathfrak{m}\) is an infinite cardinal. Then the union \(\bigcup_\alpha A_\alpha\)
  has size at most \(\mathfrak{m}\).
\end{proposition}
\begin{proof}
  TODO
\end{proof}
